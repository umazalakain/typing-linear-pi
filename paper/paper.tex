\documentclass{scrartcl}

\usepackage[utf8]{inputenc}
\usepackage{url}
\usepackage{todonotes}
\usepackage{enumitem}
\setlist{parsep=0pt,listparindent=\parindent}
\usepackage{amssymb}
\usepackage{amsmath}
\usepackage[many]{tcolorbox}

% Links and their colors
\usepackage[
  colorlinks=true,
  linkcolor=darkgray,
  citecolor=darkgray,
  urlcolor=darkgray,
  ]{hyperref}

% Commands for the pi-calculus
\newcommand{\PO}{\mathbf{0}}
\newcommand{\comp}[2]{#1 \mid #2}
\newcommand{\new}[2]{(\boldsymbol{\nu} #1 #2)\,}
\newcommand{\cout}[2]{\overline{#1}\langle#2\rangle.}
\newcommand{\cin}[2]{#1(#2).}
\newcommand{\select}[2]{#1\triangleleft#2.}
\newcommand{\branch}[2]{#1\triangleright#2}

\newcommand{\subst}[3]{#1[#2/#3]}

\newcommand{\picalc}{$\pi$-calculus}
\newcommand{\Picalc}{$\pi$-Calculus}

\newcommand{\type}{\texttt}
\newcommand{\End}{\type{End}}
\newcommand{\Send}[1]{!#1.}
\newcommand{\Recv}[1]{?#1.}
\newcommand{\Select}{\oplus}
\newcommand{\Branch}{\&}
\newcommand{\dual}{\overline}

\newcommand{\reduce}{\rightarrow}

\begin{document}

%%%%%%%%%%%%%%%%%%%%%%%%%%%%%%%%%%%%%%%%%%%%%%%%%%%%%%%%%%%%%%%%%%%
\title{Typechecking a linear \picalc}
\author{Uma Zalakain}
\maketitle
%%%%%%%%%%%%%%%%%%%%%%%%%%%%%%%%%%%%%%%%%%%%%%%%%%%%%%%%%%%%%%%%%%%

\begin{abstract}
  We present the syntax, operational semantics, and typing rules for a \picalc{} with linear and shared types. We use a typing with leftovers approach \cite{} to encode our typing rules. We build framing and weakening proofs that we then use to prove subject congruence and subject reduction.
\end{abstract}

%%%%%%%%%%%%%%%%%%%%%%%%%%%%%%%%%%%%%%%%%%%%%%%%%%%%%%%%%%%%%%%%%%%
\section{Introduction}

Goals: present nice model, prove subject reduction

Intensional (correct by construction) vs extensional

same syntax and semantics, different typing rules

full formalisation available in Agda

%%%%%%%%%%%%%%%%%%%%%%%%%%%%%%%%%%%%%%%%%%%%%%%%%%%%%%%%%%%%%%%%%%%
\section{Related work}

\cite{previous-work} polymorphic tokens, HOAS

\cite{typing-with-leftovers}

%%%%%%%%%%%%%%%%%%%%%%%%%%%%%%%%%%%%%%%%%%%%%%%%%%%%%%%%%%%%%%%%%%%
\section{Syntax}

variable references (strings, locally named, de Bruijn)

strings to maybe de Bruijn

%%%%%%%%%%%%%%%%%%%%%%%%%%%%%%%%%%%%%%%%%%%%%%%%%%%%%%%%%%%%%%%%%%%
\section{Semantics}

keeping track of the variable on which communication occurs

%%%%%%%%%%%%%%%%%%%%%%%%%%%%%%%%%%%%%%%%%%%%%%%%%%%%%%%%%%%%%%%%%%%
\section{Linear typing rules}

using dependent types to encode typing rules

removing from context vs keeping in context but marking it used

two-layered approach: types on one hand, capabilities on the other

n multiplicities as a generalisation

properties of the underlaying monoid

nat numbers + omega as a generalisation

partial vs total

\subsection{Typing with leftovers}

Variable references as proofs of capability

Context splits at each variable reference

%%%%%%%%%%%%%%%%%%%%%%%%%%%%%%%%%%%%%%%%%%%%%%%%%%%%%%%%%%%%%%%%%%%
\section{Subject reduction}

\subsection{Framing}
\subsection{Weakening}

%%%%%%%%%%%%%%%%%%%%%%%%%%%%%%%%%%%%%%%%%%%%%%%%%%%%%%%%%%%%%%%%%%%
\section{Future work}

Work that will be done time permiting:

Decidable typechecking

Proof of progress

Product types

Sum types

Encoding of session types


\footnotesize
\bibliographystyle{alpha}
\bibliography{paper}
\end{document}
